\documentclass[onecolumn, draftclsnofoot,10pt, compsoc]{article}
\usepackage{graphicx}
\usepackage{url}
\usepackage{lscape}
\usepackage{setspace}
\usepackage{parskip}
\usepackage{geometry}
\geometry{textheight=9.5in, textwidth=7in}

% 1. Fill in these details
\def \CapstoneTeamName{AgBizClimate}
\def \CapstoneTeamNumber{26}
\def \GroupMemberOne{	Thomas Noelcke}
\def \GroupMemberTwo{	Shane Barrantes}
\def \GroupMemberThree{	Shengpei Yuan}
\def \CapstoneProjectName{ Linking Seasonal Weather Data to AgBizClimate\texttrademark}
\def \CapstoneSponsorCompany{ Oregon State University}
\def \CapstoneSponsorPerson{ Clark Seavert}

% 2. Uncomment the appropriate line below so that the document type works
\def \DocType{		%Software Requirements Document
				%Requirements Document
				%Technology Review
				%Design Document
				Mid Term Progress Report Winter 2018 
				}

\newcommand{\NameSigPair}[1]{\par
\makebox[2.75in][r]{#1} \hfil 	\makebox[3.25in]{\makebox[2.25in]{\hrulefill} \hfill		\makebox[.75in]{\hrulefill}}
\par\vspace{-12pt} \textit{\tiny\noindent
\makebox[2.75in]{} \hfil		\makebox[3.25in]{\makebox[2.25in][r]{Signature} \hfill	\makebox[.75in][r]{Date}}}}
% 3. If the document is not to be signed, uncomment the RENEWcommand below
\renewcommand{\NameSigPair}[1]{#1}

%%%%%%%%%%%%%%%%%%%%%%%%%%%%%%%%%%%%%%%
\begin{document}
\begin{titlepage}
    \pagenumbering{gobble}
    \begin{singlespace}
        \hfill
        % 4. If you have a logo, use this includegraphics command to put it on the coversheet.
        %\includegraphics[height=4cm]{CompanyLogo}
        \par\vspace{.2in}
        \centering
        \scshape{
            \huge CS Capstone \DocType \par
            {\large\today}\par
            \vspace{.5in}
            \textbf{\Huge\CapstoneProjectName}\par
            \vfill
            {\large Prepared for}\par
            \Huge \CapstoneSponsorCompany\par
            \vspace{5pt}
            {\Large\NameSigPair{\CapstoneSponsorPerson}\par}
            {\large Prepared by }\par
            Group\CapstoneTeamNumber\par
            % 5. comment out the line below this one if you do not wish to name your team
            %\CapstoneTeamName\par
            \vspace{5pt}
            {\Large
                \NameSigPair{\GroupMemberOne}\par
                \NameSigPair{\GroupMemberTwo}\par
                \NameSigPair{\GroupMemberThree}\par
            }
            \vspace{20pt}
        }
        \begin{abstract}
						
        \end{abstract}
    \end{singlespace}
\end{titlepage}
\newpage
\pagenumbering{arabic}
\tableofcontents
% 7. uncomment this (if applicable). Consider adding a page break.
\listoffigures 
\newpage
%\listoftables
\clearpage

% 8. now you write!
%this section can likely be coppied from the design doc.
\section{Introduction}
	\subsection{Purpose}
		The purpose of this document is to describe the progress we have made so far on the \textit{AgBizClimate} project. In this document we will give a brief introduction to the \textit{AgBizCliamte} project. In this section we will discuss the purpose of the project. Additionally, we will also discuss the scope of the project and an overview of the project functions.\\
		This document is designed for the project owners. This document is also designed for the development team so we can evaluate our progress so far on this project. This project is also designed to fulfill the minimum requirements for the CS461 class for the OSU computer science program.\\
		
				\subsection{Overview}
			Seasonal climate is one of the essential factors that affects agricultural production. As a module of \textit{AgBiz Logic}, \textit{AgBizClimate} delivers essential information about climate change to farmers, and help professionals to develop management pathways that best fit their operations under a changing climate. This project aims to link the crucial seasonal climate data from the Northwest Climate Toolbox database to \textit{AgBiz Logic} so that it can provide changes in net returns of crop and livestock enterprises through powerful graphics and tables.\\

		\subsection{Scope}
			This project is a part of a much larger AgBiz Logic\texttrademark program. However, the purpose of this project is to add a short term climate tool to the \textit{AgBizClimate} module. This limits the scope of the project to the \textit{AgBizClimate} Module. Additionally, we will only be adding the short term climate data tool as the long term climate data tool already exists.\\

			Currently \textit{AgBizClimate} has a long-term climate tool but no such tool exists for short term climate data. We will implement a tool to extract short-term climate data from the Northwest Climate Toolbox database, display it to the user and allow the user to adjust crop and livestock yields or quality of products sold and, production inputs. Moreover, a landing tool will be developed to allow users to switch between short-term seasonal tool and long-term climate data tool.\\
			
		\subsection{Definitions, Acronyms and Abbreviations}
			REST - Representational State Transfer, This is a type of architecture that manages the state of the program. This is especially popular in web development.\\
			API- Application Programming Interface. This is a piece of software that allows a connection to another piece of software providing some sort of service.\\
			NWCTB - Northwest Climate Toolbox. This is the database we will be connecting to that will provide the short term climate data we plan to use.\\
			Thredds Data Server - This is a web server that provides meta-data and data access for scientific data sets using OPeNDAP along with some other remote data access protocols.\\
			OPeNDAP - Open-source Project for a Network Data Access Protocol. This is the protocol we will be using to retrieve the data sets from the Thredds data server.\\
			NMME - North American Multi-Model Ensemble. This is a data set that brings together a variety of different weather models into one data set.\\
			Climate Scenario - This is a theoretical calculation of yields, inputs and of the overall budget for one situation based on the climate data.\\
			NETCDF - This is a file storage format for large scientific data sets especially good for any data that is referenced on a grid and related to is geo-location.\\
			
		
		%will need updates.
		\subsection{Product Function Overview}
		    \textit{AgBizClimate} is a web based decision tool that will allow users to gain specific insight on how localized climate data for the next seven months will affect their crop and livestock yields or quality of products sold and production inputs. The \textit{AgBizClimate} tool will allow users to input their location (state, county) and a budget for the specific crop or livestock enterprise. \textit{AgBizClimate} will select climate data for the next seven months for that location and provide graphical data showing temperature and precipitation. Users will then be able to change yields or quality of product sold by a percentage they think these factors will affect and modify production inputs. Finally the tool will calculate the net returns.\\
				

		    
		\renewcommand\refname{\vskip -1cm}
		\subsection{References}

		    \nocite{*}
            \bibliographystyle{IEEEtran}
            \bibliography{IEEEabrv,References}
            

\section{Current Project State}
    In this section of this document we will review what items we have resolved, List and explain what items we still have to do, discuss major blockers and give a week by week summary of what we have done so far. This section is intended to give a good overall picture of the status of the \textit{AgBizClimate} project along with what we have accomplished so far.\\

	\subsection{Resolved Items}
		\paragraph{Updated Requirements Document:} The requirements document has been updated to reflect changes to the climate data API. We were expecting to have an endpoint we could call with a latitude and a longitude however accessing the climate data is going to require more work that we had originally expected. So we've updated the requirements document to reflect this change.\\
		
		
		\paragraph{Create Concept script for Thredds database:} We've created a concept script to use as an example in our climate API that connects to the Thredds database and read back the climate data for the Latitude and Longitude we've request. However, we are getting errors on reads latter in the data set that we can't explain.\\
		
		
		\paragraph{Install Script for NETCDF4 and Dependencies:} Another problem we encountered while creating the concept script was setting up and installing the python NETCDF4 library. Luckily we were able to write a scrip that installs this library and its dependencies.\\
		
		\paragarph{Added option to select Short term Climate Scenario:} We added a drop down to the region selection page that allows users to select short term climate scenarios and long term climate Scenarios. This will allow people to use the tool we are creating.\\
		
		%this isn't actually true yet but I'm working on it and getting close.
		\paragraph{Created The Charts Page:} We've also created the page where the users will view the climate data. This page will also allow the users to enter a change in yield for their crops and will then redirect them to the Budget review page. Though this page has a few issues most of the work is completed.\\
		
		
	\subsection{In progress}
	In this section we will discuss work items that are in active development. We will describe each item giving a brief description. We will also discuss how much progress has been made on each item.\\
		
		\paragraph{Climate Data API:} We have started work on the back end Climate data API however currently it is just an end point that distributes mocked data as we have run into issues with the netcdf4 library we are using to connect to the database with all the data in it. Once we have a solution for connecting to and reading all the values from the database it won't be hard to get real data from the API.\\
		
		\paragraph{Document Updates:} We have had some serious changes to the expected design as a result we will need to make some document updates. We have started some of these updates. However, I don't anticipate being able to finish all the document updates this term.\\
	        
	\subsection{ToDo's}
	    In this section we will discuss items we will need still need to do over the next two terms. We will give a brief description of each item. We will also state an approximate time for when we expect each item to be completed.\\
			
			\paragraph{Backend Testing:} Given that our Climate data has not yet been well defined we haven't started writing test to test the back end code. Once we get the API written in something close to its final form we will want to start writing some test to ensure that the API meets the requirements in the requirements document.\\
			
			\paragraph{Front End testing:} Similarly as the front end work is still in active development we haven’t written any tests to cover this code. However we will be done with front end development soon and should be able to get to work doing front end testing.\\
			
			\paragraph{Finding a solution to NETCDF4 Problems:} We will need to find a work around to get around errors we are having while reading from the database using the netcdf4 library. Currently, we think we can get around this by writing the bit that communicates with the server in c\+\+. However we are trying installing dependencies for the netcdf library in a different way first.\\
			
			\paragraph{QA:} Finally, before we are done with our project will want to provide some QA on the work that we will have done over the term. This will include running some manual integration tests to ensure that we have completed the work we agreed to complete.\\
	    
	\subsection{Blockers}
	    In this section we will discuss hurtles we are facing that are impeding progress on this project. We will discuss why they will keep us from moving forward on this project and we will also discuss what we intend to do to move past these problems.\\
			
			\subsubsection{netcdf problems}
			Initially, we had issues getting netcdf4, a python library for dealing with the data files that contain the 
	
	%will copy from one note. Will need to modify weekly summary from each week just a tiny bit to make this section work    
\section{Weekly summary of progress}
	   In this section we will give a weekly summary of our progress on this project. For each we will list out our plans, problems we have encountered during the week and will show a summery of what we have accomplished during the week.\\
			
		\subsection{Week 1}
			\subsubsection{Plans}
				\begin{itemize}
					\item Meet with group to set up iteration one of project development.
					\item Meet with Sean to set up git branch and discuss git workflow.
					\item set taks for iteration 1.
				\end{itemize}
			
			\subsubsection{Progress}
				\begin{itemize}
					\item Forked github repo from AgBiz-Logic
					\item Set up a meeting with Sean to discuss project development.
					\item Started setting up tasks for iteration one on the git repo.
					\item Started working on the Wiki page with common help items for the project.
				\end{itemize}
			\subsubsection{Problems}
				\begin{itemize}
					\item Still haven't heard any thing from the NWCTB team regarding API access for the climate data.
				\end{itemize}
				
			\subsubsection{Summary}
			This week we tried to set up a meeting with Sean to do some project planning and set up for iteration one of our project. However, Sean was unavailable this week so we set up a meeting for next week. We also got our git repo, forked from \textit{AgBiz-Logic} set up. We started planning the first iteration of development on the project by adding issues to the github repository. We also started compiling some help pages on the Wiki of our repo.\\
			
		\subsection{Week 2}
			\subsubsection{Plans}
				\begin{itemize}
				
				\end{itemize}
			\subsubsection{Progress}
				\begin{itemize}
				
				\end{itemize}
			\subsubsection{Problems
				\begin{itemize}
				
				\end{itemize}
			\subsubsection{Summary}
			
		\subsection{Week 3}
			\subsubsection{Plans}
				\begin{itemize}
				
				\end{itemize}
			\subsubsection{Progress}
				\begin{itemize}
				
				\end{itemize}
			\subsubsection{Problems}
				\begin{itemize}
				
				\end{itemize}
			\subsubsection{Summary}
			
		\subsection{week 4}
			\subsubsection{Plans}
				\begin{itemize}
				
				\end{itemize}
			\subsubsection{Progress}
				\begin{itemize}
				
				\end{itemize}
			\subsubsection{Problems}
				\begin{itemize}
				
				\end{itemize}
			\subsubsection{Summary}
						
		\subsection{week 5}
			\subsubsection{Plans}
				\begin{itemize}
				
				\end{itemize}
			\subsubsection{Progress}
				\begin{itemize}
				
				\end{itemize}
			\subsubsection{Problems}
				\begin{itemize}
				
				\end{itemize}
			\subsubsection{Summary}	
					
		\subsection{week 6}
			\subsubsection{Plans}
				\begin{itemize}
				
				\end{itemize}
			\subsubsection{Progress}
				\begin{itemize}
				
				\end{itemize}
			\subsubsection{Problems}
				\begin{itemize}
				
				\end{itemize}
			\subsubsection{Summary}

\section{Individual Section}
In this section we will each write about our individual parts of the project and the progress we have made. We will also use this section to provide a peer review for our group mates rating their performance and contribution to the project.\\

\subsection{Thomas Noelcke}
	In this section I will discuss the various contributions I've made to the project along with some of the problems I've come across including any solutions to problems that I’ve found. I will also discuss the contribution of my team mates to this project giving each of my group mates a peer review.\\
	
	\subsubsection{My Contributions}
		In this section I will discuss my contributes to the project highlighting problems I've run into and solution that I've found.\\
		\paragraph{documentation} \hfill \break
		
		\paragraph{Concept API Script} \hfill \break
		
		\paragraph{Front-end Work} \hfill \break
		
		
		\paragraph{My Sections}
	
	\subsubsection{Peer Review}
		In this section I will provide a peer review for each of my group mates. For each group mate I will describe contributions made by each group member pointing out things each group member does well 
		\paragraph{Shane Barrantes} \hfill \break
			Throughout this project this term Shane has been a helpful team mate and has contributed to the project both by writing feature code and by preforming complex environment setup. Shane has provided much input into the execution and planing of this project throughout the beginning of the term. He has been an above average team mate in showing up to and contributing to meetings and work sessions.\\ 
			
			I will be first to admit that package management and environment setup are areas of software development that I am weak in. As a result early in this project I struggled getting the environment set up and running. Shane was a huge help in setting up the environment and getting the project running. Not only did Shane help me fix issues with getting the project to build, he also taught me about package managers and project set up in general. Shane also helped the team out by writing install scripts for complicated installs. For instance early in the project we ran into many issues installing the Python netcdf4 library. Shane figured out how to install it and wrote script that installed the netcdf4 library along with its dependencies. Shane has also been very helpful in regards to developing the frond end and back end code. Shane started developing a complicated charts page on the front. He helped get the page set up so that the front end development could be carried out. However, he never had the chance to get that page completed because I needed him to create an elaborate install script that installed the HDF5 library from source enabling certain options on that package.\\
			
						The only criticism that I may provide on Shane's performance so far this term is that outside of group meetings he hasn't been able to put in as much time as I had hoped. Generally, I think this has been difficult for Shane due a few things beyond his control such as deadlines at work and a heavy course load. As this project progresses this is one area where I hope to see improvement.\\
		
			
			Shane has also shown his value as a team mate in the way that he communicates and contributes to meetings. Shane is not able to make it to every meeting, however Shane is very good about communicating when he is unable to make it to a meeting and does a good job of warning me of potential scheduling conflicts. When Shane is in meetings he is present, listening and contributing to the conversation. Shane generally focuses well on his work when at work sessions and group meetings.\\
			
			Overall I would give Shane 8\/10 as a group mate. I've taken two points off because Shane has had a hard time contributing to the project outside of group meetings and work sessions. I anticipate that this will improve as Shane has expressed that he is close to finishing a big project at work. Shane as also expressed that he would also like to see this aspect of his performance Improve.\\
			
		\paragraph{Shengpei Yuan} \hfill \break
		
		

\subsection{Shane Barrantes}

\subsection{Shengpei Yuan}


\end{document}

