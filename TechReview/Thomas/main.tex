\documentclass[letterpaper,10pt]{article}

\usepackage{graphicx}
\usepackage{amssymb}
\usepackage{amsmath}
\usepackage{amsthm}

\usepackage{alltt}
\usepackage{float}
\usepackage{color}
\usepackage{url}

\usepackage[TABBOTCAP, tight]{}
\usepackage{enumitem}

\usepackage{geometry}
\geometry{textheight=8.5in, textwidth=6in}

%random comment

\newcommand{\cred}[1]{{\color{red}#1}}
\newcommand{\cblue}[1]{{\color{blue}#1}}

\usepackage{hyperref}
\usepackage{geometry}

\begin{document}
    \begin{titlepage}
    \newcommand{\HRule}{\rule{\linewidth}{0.5mm}}
    \center
    \textsc{\Large Oregon State University}\\[1.5cm]
    \textsc{\Large CS 461}\\[0.5cm]
    \textsc{\Large Fall 2017}\\[0.5cm]
    \HRule \\[0.4cm]
    { \huge \bfseries Tech Review}\\[0.4cm] % Title of your document
    \HRule \\[1.5cm]
    \begin{minipage}{0.4\textwidth}
        \begin{flushleft} \large
        \emph{Author:}\\
        Thomas Noelcke
        \end{flushleft}
    \end{minipage}
    \begin{minipage}{0.4\textwidth}
        \begin{flushright} \large
        \emph{Instructor:} \\
        D. Kevin McGrath\\
        Kirsten Winters
        \end{flushright}
    \end{minipage}\\[2cm]
    \begin{abstract}
    \item 
		The pourpose of this document is to research and consider different technical options for our applcaiton. In this document we research different options for data storage, HTTP request frame works, and testing frameworks. I will consider three possible choices for each section of the application. For each of these options I will weight the pros and cons of each. After comparing the different options I will select the option I would like to use for the AgBizClimate application.
    \end{abstract}
    \vfill % Fill the rest of the page with whitespace
    \end{titlepage}
		
\section{Data Storage}
	\subsection{Overview}
		For the AgBizClimate applicatoin we will need a way to store data so it can be easily retervied later. For this application we will store a variaty of data including budget data, weather data, and user information. This information will need to be quickily recalled so it can be used in our application. Generally, we will want to select a data storage option that will be easy to set up and allow us alot of flexiblity with what kind of data can be stored. We will also want a data storeage option that will quickly recall stored data so it can be used by the application.\\
		To analyize our options for our application we will define Criteria by which to compare the different options. We will then use the critiria to compare the three options for our application. Once we have compared our options we will select which opiton or options we plan to use for our application. We will then justify why we choose that option. If we chose multiple options we will describe which situation we will use one option over the other.
		
	\subsection{Criteria}
	\subsection{Potential Choices}
		\subsubsection{NoSQL}
		
		\subsubsection{MongoDB}
		
		\subsubsection{}
		
	\subsection{Discussion}
	
	\subsection{Conclusion}
	
	
\section{Item 2}
		\subsection{Overview}
	
	\subsection{Criteria}
	
	\subsection{Potential Choices}
		\subsubsection{Choice 1}
		
		\subsubsection{Choice 2}
		
		\subsubsection{choice 3}
		
	\subsection{Discussion}
	
	\subsection{Conclusion}
	
	
\section{Item 3}
		\subsection{Overview}
	
	\subsection{Criteria}
	
	\subsection{Potential Choices}
		\subsubsection{Choice 1}
		
		\subsubsection{Choice 2}
		
		\subsubsection{choice 3}
		
	\subsection{Discussion}
	
	\subsection{Conclusion}
	
\section{References}


\end{document}