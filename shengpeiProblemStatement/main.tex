\documentclass[letterpaper,10pt]{article}

\usepackage{graphicx}
\usepackage{amssymb}
\usepackage{amsmath}
\usepackage{amsthm}

\usepackage{alltt}
\usepackage{float}
\usepackage{color}
\usepackage{url}
\usepackage{indentfirst}

\usepackage[TABBOTCAP, tight]{}
\usepackage{enumitem}

\usepackage{geometry}
\geometry{textheight=8.5in, textwidth=6in}

%random comment

\newcommand{\cred}[1]{{\color{red}#1}}
\newcommand{\cblue}[1]{{\color{blue}#1}}

\usepackage{hyperref}
\usepackage{geometry}

\begin{document}
    \begin{titlepage}
    \newcommand{\HRule}{\rule{\linewidth}{0.5mm}}
    \center
    \textsc{\Large Oregon State University}\\[1.5cm]
    \textsc{\Large CS 461}\\[0.5cm]
    \textsc{\Large Fall 2017}\\[0.5cm]
    \HRule \\[0.4cm]
    { \huge \bfseries Problem Statement}\\[0.4cm] % Title of your document
    \HRule \\[1.5cm]
    \begin{minipage}{0.4\textwidth}
        \begin{flushleft} \large
        \emph{Author:}\\
        Shengpei Yuan
        \end{flushleft}
    \end{minipage}
    \begin{minipage}{0.4\textwidth}
        \begin{flushright} \large
        \emph{Instructor:} \\
        D. Kevin McGrath\\
        Kirsten Winters
        \end{flushright}
    \end{minipage}\\[2cm]
    \begin{abstract}
    \item The seasonal climate change is one of the essential factors for the harvest and returns of crops and farming investment programs of enterprises and organizations. As a sub-project of AgBizLogic, AgBizClimate is dedicated to deliver essential information about climate change to farmers, and help professionals to develop management pathways that best fit their operations under a changing climate. This project aims to link the crucial seasonal climate data from the Northwest Climate Toolbox database to AgBizLogic so that it could make specific analysis and demonstrating through powerful graphics. AgBixClimate is designed to enable farmers and agriculture enterprises to decide appropriate farming investment projects for their crops and products.\\
    Currently AgBizClimate has a long-term climate tool but no such tool exists for short term climate data. We will implement a tool to extract short-term climate data from the Northwest Climate Toolbox database. Moreover, a landing tool will be developed to allow users to switch between short term seasonal tool and long-term climate data tool.

    \end{abstract}
    \vfill % Fill the rest of the page with whitespace
    \end{titlepage}

    \section*{Definition and description of the problem}
    The seasonal weather data is important for farmers and land managers as climate may have great impacts on crops. For instance, the precipitation on different days across the life cycles of crops may have different impacts on the harvest. Land managers and farmers used to rely on past experiences of climate data from the past to make decisions for specific farming operations in order to reduce the negative impacts or make use of favorable climate conditions. However, these individual experiences of weather data are often limited and inaccurate. Professional tools like software systems could be adopted to build models for decision making based on available seasonal climate data.\\
    
    AgBizClimate is a sub-project of AgBiz Logic (https://www.agbizlogic.com) that works on helping farmers and ranchers improve profits by providing constructive advice for decisions on investments and programs. AgBizClimate is designed to deliver essential information about climate change to farmers and land managers for specific farming operations. However, we have to collect sufficient climate data before making any analysis. Therefore, the aim of this project is to link seasonal weather data from somewhere to AgBizClimate. Specifically, we plan to transfer and integrate weather data from the Northwest Climate Toolbox database (https://climatetoolbox.org) into AgBizClimate (https://www.agbizlogic.com). We will then use this data for analysis and presentation of the influence of climate change on costs and returns of farming programs.\\
    
    Generally, the tool is designed to work as follows. First, there are various crops that are supported to be examined in regards to  relevant seasonal climate data. Users can choose three to five different crops for analysis. Secondly, users can place a pin on a map to view the objective climate data and crops for that location. Thirdly, the climate data of the specified location is plotted and displayed. Lastly, users would be able to adjust the crops yields, inputs and prices by simple GUI operations and view the expected returns immediately.\\
    
    Although it sounds simple, there are several key issues one needs to overcome before successfully demonstrating data by figures on AgBizClimate web pages. Firstly, the weather data could vary a lot in terms of both formats and dimensions on the original Northwest Climate Toolbox database, and one has to choose appropriate formats and necessary dimensions according to the application contexts. Secondly, the dynamically collected data should be appropriately stored and managed so that they could be flexibly employed in AgBizClimate. Thirdly, one need to choose suitable styles and relevant map tools capable of extracting and demonstrating seasonal climate data in order to help users understand the underlying impacts of them on harvest and returns of farming programs according to climate change.

	  \section*{Proposed solution}
   As we have decided the source of the seasonal climate data is Northwest Climate Toolbox database (https://climatetoolbox.org) that provides professional climate data of Northwest area with various formats, we need a powerful tool to retrieve them from remote database server and save them in the local AgBizClimate server. And then we could parse the local climate data, and extract interested parts or make transformations according to the practical application scenarios. Finally, the desired climate data are provided to the AgBizClimate applications for further analysis or demonstrating purposes.\\
   
   The main tool we choose is Python programming language, which is lightweight and efficient. It is good at network programming, parsing and transforming various formats of data like XML and JSON.\\
   
   For the three main issues mentioned in the previous part, we plan to solve them with following solutions. First, the XML or JSON formats of raw data from the climate toolbox are favored as we could easily handle them by Python, and specific fields names are predefined in configuration files for reducing dimensions of climate data. Secondly, database tools like sqlite and MongoDB would be adapted to efficiently store and mange the retrieved climate data. And it is easy for Python to operate on these databases. Thirdly, for the various tools capable of extracting and demonstrating climate data by colorful graphic depictions with map on web pages, we are in favor of open source frameworks like python Django and AngularJS.

	  \section*{Performance metrics}
    It is helpful to give some general metrics on the completeness of the project so that the clients could have a basic conception of the products of the project. Overall, we will get actionable climate data that can be integrated into the AgBizClimate for analysis and presentations. More specifically, we have following basic evaluation criteria:\\
    
    First, interested climate data for crops could be retrieved through network (Internet) by programs (scripts) automatically without manual intervention under normal conditions, and then it could be displayed on a map and easily navigated. That is, once configured, the programs (scripts) should be able to run and download desired climate data from remote climatetoolbox server automatically.\\
    
    Second, it is known that the climate data is dynamically increased as time goes on. Therefore, climate data should be able to be updated periodically or triggered manually by clicking a button or typing a command on the local programs. And there should be configuration files that define the parameters like time interval, etc.\\
    
    Thirdly, it is possible that users want to examine the original climate data downloaded from remote climatedtoolbox server. Thus the raw climate data should be able to be exported easily in flat file formats like csv or txt. Moreover, when exporting data, it is required that multiple filtering or query conditions like dates and specific fields (dimensions) could be designated for the climate data as the entire data could be too large.\\
    
    Last but not least, these operations should be friendly to both professional and non-professional users and always give users warnings or tips if they are making unsafe operations like deleting data files.




\end{document}
