\documentclass[letterpaper,10pt]{article}

\usepackage{graphicx}
\usepackage{amssymb}
\usepackage{amsmath}
\usepackage{amsthm}

\usepackage{alltt}
\usepackage{float}
\usepackage{color}
\usepackage{url}

\usepackage[TABBOTCAP, tight]{}
\usepackage{enumitem}

\usepackage{geometry}
\geometry{textheight=8.5in, textwidth=6in}

%random comment

\newcommand{\cred}[1]{{\color{red}#1}}
\newcommand{\cblue}[1]{{\color{blue}#1}}

\usepackage{hyperref}
\usepackage{geometry}

\begin{document}
    \begin{titlepage}
    \newcommand{\HRule}{\rule{\linewidth}{0.5mm}}
    \center
    \textsc{\Large Oregon State University}\\[1.5cm]
    \textsc{\Large CS 461}\\[0.5cm]
    \textsc{\Large Fall 2017}\\[0.5cm]
    \HRule \\[0.4cm]
    { \huge \bfseries Problem Statement}\\[0.4cm] % Title of your document
    \HRule \\[1.5cm]
    \begin{minipage}{0.4\textwidth}
        \begin{flushleft} \large
        \emph{Author:}\\
        Shane Barrantes
        \end{flushleft}
    \end{minipage}
    \begin{minipage}{0.4\textwidth}
        \begin{flushright} \large
        \emph{Instructor:} \\
        D. Kevin McGrath\\
        Kirsten Winters
        \end{flushright}
    \end{minipage}\\[2cm]
    \begin{abstract}
    \item 
Oregon State University’s Department of Applied Economics has produced a mobile app named AgBizLogic to help local businesses gauge investments surrounding the agriculture industry. We have been given the task to produce a new submodule for the AgBizLogic tool named AgBizClimate. The AgBizClimate product will be a mobile app that integrates with the rest of AgBizLogic toolset that takes user input and provide analysis, visualization, and predictions based on local climate data back to the user.

The primary challenges from this project will be integrating ourselves with the AgBiz team’s code base and stack and utilizing the Northwest Climate Toolbox database to create relevant and useful output from users input.

    \end{abstract}
    \vfill % Fill the rest of the page with whitespace
    \end{titlepage}
(I feel like it’s necessary to mention that we have not met with our client at the time of writing this so most of this post is guess work and feels like rambling with no real structure to fulfill the basic requirements. I feel this is not a good representation of my writing ability.)

The Oregon State University Applied Economics Department has created a modular tool named agBizLogic to help local businesses make smart decisions in investing in local agriculture and gauge their investments profitability. Our team has been selected to create a submodule for the AgBizLogic project named AgBizClimate. The AgBizClimate module currently does not exist so we will be building it from scratch.

The AgBizLogic project is a mobile app available on the internet via phone, tablet, or computer. Some of the submodules for the AgBizLogic project are paid for and some are available for free. The primary goal for the AgBizLogic product is to provide provide analysis, visualization, and climate data based off of parameters provided by the user. The AgBizClimate module will focus on the climate side of visual analysis.

Due to the AgBizLogic project already being fairly far in development we have been given a list of languages that we must use to produce the AgBizClimate submodule. The primary language required for this task is Python becaue the back end of the AgBizLogic product is written in the web framework Django. The front end tools and languages required for this task are Nodejs and Angularjs. We will be fully integrating ourselves with the AgBiz team’s stack and will be utilizing github to version control our module alongside the rest of the team. We will be regularly be interacting with the AgBiz APIs calls to interact with and integrate the AgBizClimate submodule into the rest of the AgBizLogic Project. FInally we will be utilizing the Northwest Climate Toolbox database as the primary resource for data to produce the AgBizClimate tool that will provide users with relevant data based on their given parameters. 

The primary required language for the s Python due to the fact that the mobile app is written in the web framework Django. The front end languages required for this task are nodejs and angularjs. The version control system our team will be using is Git to link our work with the rest of the AgBiz team. Finally we will be using JSON to move the data from the database to displaying it on the mobile app. Our team will be interacting with their the AgBiz rest API and the Northwest Climate Toolbox database to integrate and create a front facing web tool with a fully functional backend.

The primary problem we will face is turning the Northwest Climate Toolbox database information into a usable product by integrating and interpreting its data into something users can use to base investments off of. These figures and analysis have to extremely accurate for the product to achieve it’s purpose and to not be a liability to the AgBiz team. Once we create usable and dynamic figures we then have to integrate this submodule into the already create AgBizLogic tool utilizing their existing codebase utilizing the previously mentioned stack.

Our proposed solution is to gain fully integrate ourselves into the previously defined stack so that integration and future maintenance of the AgBizClimate product we create will be easy for the AgBizLogic team once we finish this project. These tools have been selected for this specific purpose and we have evaluated them as being apt tools for the task at hand.

The performance metrics for this tool are difficult to gauge at this time due to a lack of information, however we can extrapolate what they will be. Since the AgBizClimate product does not currently exist as a functioning front facing web tool that is a piece of the AgBizLogic tool our end game is to do exactly that. This consists of create a front facing subsection of the mobile app that can intake user parameters. A backend that will interact with the front end to make the necessary calls to the used relational database to grab the necessary information and provide visual and analytic predictions based on climate and time of year back to the user. Essentially we can to use the stack to create a modeling tool to further AgBizLogic toolchain under the name AgBizClimate. When discussing the frontend part of this project we mean the front facing webpage that users see when interacting with AgBizClimate product we will create. The backend in this project in the piece that is hidden from users that will interact with the necessary information to produce visual information to the users that can be useful to them and assist in mobile application navigation. Since we will be working with something that does not currently exist these are the primary metrics that we we will use to gauge whether we are done with the product. (More actual specifics will be given once we have our meeting with the client). In order for our team to be satisfied with the outcome of this project we expect to have working standalone module that is easily integratable. We gauge standalone as being a web app that can function without the rest of the AgBizLogic product excluding the database. The term easily integratable refers to the usage of their stack in it’s entirety for the relevant parts of the project. This will allow the code to be easily portable to other areas in the AgBizLogic project.

Our primary performance goals for this project consist of the AgBizClimate module as being usable, defined as each task taking no longer than three seconds from when any action is started with initial user feedback starting within a quarter of a second after user interaction is given.


\end{document}
