\documentclass[onecolumn, draftclsnofoot,10pt, compsoc]{article}
\usepackage{graphicx}
\usepackage{url}
\usepackage{lscape}
\usepackage{setspace}
\usepackage{parskip}
\usepackage{geometry}
\geometry{textheight=9.5in, textwidth=7in}

% 1. Fill in these details
\def \CapstoneTeamName{AgBizClimate}
\def \CapstoneTeamNumber{26}
\def \GroupMemberOne{	Thomas Noelcke}
\def \GroupMemberTwo{	Shane Barrantes}
\def \GroupMemberThree{	Shengpei Yuan}
\def \CapstoneProjectName{ Linking Seasonal Weather Data to AgBizClimate\texttrademark}
\def \CapstoneSponsorCompany{ Oregon State University}
\def \CapstoneSponsorPerson{ Clark Seavert}

% 2. Uncomment the appropriate line below so that the document type works
\def \DocType{		%Software Requirements Document
				%Requirements Document
				%Technology Review
				%Design Document
				Progress Report
				}

\newcommand{\NameSigPair}[1]{\par
\makebox[2.75in][r]{#1} \hfil 	\makebox[3.25in]{\makebox[2.25in]{\hrulefill} \hfill		\makebox[.75in]{\hrulefill}}
\par\vspace{-12pt} \textit{\tiny\noindent
\makebox[2.75in]{} \hfil		\makebox[3.25in]{\makebox[2.25in][r]{Signature} \hfill	\makebox[.75in][r]{Date}}}}
% 3. If the document is not to be signed, uncomment the RENEWcommand below
\renewcommand{\NameSigPair}[1]{#1}

%%%%%%%%%%%%%%%%%%%%%%%%%%%%%%%%%%%%%%%
\begin{document}
\begin{titlepage}
    \pagenumbering{gobble}
    \begin{singlespace}
        \hfill
        % 4. If you have a logo, use this includegraphics command to put it on the coversheet.
        %\includegraphics[height=4cm]{CompanyLogo}
        \par\vspace{.2in}
        \centering
        \scshape{
            \huge CS Capstone \DocType \par
            {\large\today}\par
            \vspace{.5in}
            \textbf{\Huge\CapstoneProjectName}\par
            \vfill
            {\large Prepared for}\par
            \Huge \CapstoneSponsorCompany\par
            \vspace{5pt}
            {\Large\NameSigPair{\CapstoneSponsorPerson}\par}
            {\large Prepared by }\par
            Group\CapstoneTeamNumber\par
            % 5. comment out the line below this one if you do not wish to name your team
            %\CapstoneTeamName\par
            \vspace{5pt}
            {\Large
                \NameSigPair{\GroupMemberOne}\par
                \NameSigPair{\GroupMemberTwo}\par
                \NameSigPair{\GroupMemberThree}\par
            }
            \vspace{20pt}
        }
        \begin{abstract}
					In this document we will give an overview of the progress we have made this term on the \textit{AgBizClimate} project. In this document we will begin giving a brief introduction to the project. In this section we will discuss the purpose of the project, the scope of the project and give an overview of the project function. Then we will discuss the current state of the project. In this section we will talk about what we have done, what we have to do and show a week by week summery of progress. Next we will discuss items that are blocking our progress moving forward. Finally we will give a Retrospective of the last ten weeks.\\
        \end{abstract}
    \end{singlespace}
\end{titlepage}
\newpage
\pagenumbering{arabic}
\tableofcontents
% 7. uncomment this (if applicable). Consider adding a page break.
\listoffigures 
\newpage
%\listoftables
\clearpage

% 8. now you write!
%this section can likely be coppied from the design doc.
\section{Introduction}
	\subsection{Purpose}
		The purpose of this document is to describe the progress we have made so far on the \textit{AgBizClimate} project. In this document we will give a brief introduction to the \textit{AgBizCliamte} project. In this section we will discuss the purpose of the project. Additionally, we will also discuss the scope of the project and an overview of the project functions.\\
		This document is designed for the project owners. This document is also designed for the development team so we can evaluate our progress so far on this project. This project is also designed to fulfill the minimum requirements for the CS461 class for the OSU computer science program.\\
		
				\subsection{Overview}
			Seasonal climate is one of the essential factors that affects agricultural production. As a module of \textit{AgBiz Logic}, \textit{AgBizClimate} delivers essential information about climate change to farmers, and help professionals to develop management pathways that best fit their operations under a changing climate. This project aims to link the crucial seasonal climate data from the Northwest Climate Toolbox database to \textit{AgBiz Logic} so that it can provide changes in net returns of crop and livestock enterprises through powerful graphics and tables.\\

		\subsection{Scope}
			This project is a part of a much larger AgBiz Logic\texttrademark program. However, the purpose of this project is to add a short term climate tool to the \textit{AgBizClimate} module. This limits the scope of the project to the \textit{AgBizClimate} Module. Additionally, we will only be adding the short term climate data tool as the long term climate data tool already exists.\\

			Currently \textit{AgBizClimate} has a long-term climate tool but no such tool exists for short term climate data. We will implement a tool to extract short-term climate data from the Northwest Climate Toolbox database, display it to the user and allow the user to adjust crop and livestock yields or quality of products sold and, production inputs. Moreover, a landing tool will be developed to allow users to switch between short-term seasonal tool and long-term climate data tool.\\
			
		\subsection{Product Function Overview}
		    \textit{AgBizClimate} is a web based decision tool that will allow users to gain specific insight on how localized climate data for the next seven months will affect their crop and livestock yields or quality of products sold and production inputs. The \textit{AgBizClimate} tool will allow users to input their location (state, county) and a budget for the specific crop or livestock enterprise. \textit{AgBizClimate} will select climate data for the next seven months for that location and provide graphical data showing temperature and precipitation. Users will then be able to change yields or quality of product sold by a percentage they think these factors will affect and modify production inputs. Finally the tool will calculate the net returns.\\
		    
		\renewcommand\refname{\vskip -1cm}
		\subsection{References}

		    \nocite{*}
            \bibliographystyle{IEEEtran}
            \bibliography{IEEEabrv,References}
            

\section{State of the Project}
    In this section of this document we will review what items we have resolved, List and explain what items we still have to do, discuss major blockers and give a week by week summary of what we have done so far. This section is intended to give a good overall picture of the status of the \textit{AgBizClimate} project along with what we have accomplished so far.\\

	\subsection{Resolved Items}
	    In this section we will discuss the items that we have completed this term. We will summarize each item and discuss what went well and what didn't go well with each item.\\
	    
	    \subsubsection{Problem Statement}
	       The problem statement was the first group assignment that we were assigned this term. In this assignment we were to take our personal problem statements and merge them together into final group document. We did this by picking one individuals group statement and using this as a base to improve upon. We chose Shengpei's document because we felt it had the correct level of detail along with a good over all structure. Generally, I think we worked effectively as a group to accomplish this task. However, the resulting problem statement Probably ended up being a bit to technical in the details we discussed. We could have done a better job focusing on the high level problem we were trying to solve. For more on the problem statement please see our problem statement.\\
	        
	    \subsubsection{Requirements Document}
	        The Requirements Document was the next piece of documentation that we completed for this project. The purpose of this document was to outline our responsibilities for this project. This document included an introduction to the project, an overview of the project functions and specific requirements. The specific requirements section include all of the functionality we need to implement in order to have completed this project. this also included UI prototypes, and performance metrics by which we would measure our system.\\
	        
	        Over all this document was some of our best work this term. I think we did a great job working together as a team to get this document completed. I also think this is the most important piece of documentation the we wrote this term as it lays out what we need to do to complete this project. The only thing I think we may have been able to do better on this project was to proof read a little better as we found a few mistakes in this document after turning it in.\\
	    
	    \subsubsection{Technical Review}
	        The technical Review was another individual assignment much like the problem statement where we were all required to write a technical review. In this document we divided our project in to different modules. We then each picked 3 modules to review. With these three modules we then picked three different technical options for each. We then discussed these three options and compared them to each other. Finally, we picked one of the options and the option we would use for this project. We then took our individual documents and merged them in to one main document.\\
	        
	        Generally speaking for our project we didn't generally think this document out as well as the problem statement or the Requirements Document. This happened because the Technical Review didn't have much bearing on our project because we had been told by our client what frameworks and system we would be using. This is because our project is an add on to an existing project where trying change what frame work we are using would involve rewriting the existing system.\\
	        
	    \subsubsection{Design Document}
	        The design document was the final peace of documentation that we completed this term. In this document we outlined the design of our entire system in detail. This document started with a general overview of our systems purpose and function. We then showed our design for our system architecture. We also defined the data we plan to use in this system. We then discuss the design for each individual component. Finally we discussed how we planed to test the system and showed how the various different components fulfilled the functional requirements.\\
	        
	        This document in my opinion is one of the more important items we have completed this term. This is because this document lays out the design for the system in a way where it makes it easy to implement. Generally, speaking I think this document was fairly well written. However, some of the sections at the end got a bit hurried as we started this document later than intended and didn't realize how dense the document would be. In the future we will need to make some edits to this document to ensure that it accurately reflects what we actually plan to build.\\
	        
	\subsection{ToDo's}
	    In this section we will discuss items we will need still need to do over the next two terms. We will give a brief description of each item. We will also state which term we expect to have these items completed.\\
	    
	    \subsubsection{Impliment Climate Data API}
	    
	    \subsubsection{Impliment UI}
	    
	    \subsubsection{Impliment Front end Controller}
	    
	    \subsubsection{Impliment Back End Controller}
	    
	    \subsubsection{Unit Testing}
	    
	    \subsubsection{QA}
	    
	    \subsubsection{Project wrap up}
	
	\subsection{Blockers}
	    \subsubsection{No Climate Data API Access}
	    
	    \subsubsection{Code Access}
	
	%will copy from one note. Will need to modify weekly summary from each week just a tiny bit to make this section work    
	\subsection{Weekly summary of progress}
	
		\subsubsection{Week 1}
		
		\subsubsection{Week 2}
		
		\subsubsection{Week 3}
		
		\subsubsection{Week 4}
		
		\subsubsection{Week 5}
		
		\subsubsection{Week 6}
		
		\subsubsection{Week 7}
		
		\subsubsection{Week 8}
		
		\subsubsection{Week 9}
		
		\subsubsection{Week 10}
		
		\subsubsection{Finals Week}
		
%this section will be a three column Positives, Deltas: Things that need to change, Action Column: things that will need to be implemented in order to create necessary changes.
\section{Retrospective}
\begin{center}
    \begin{tabular}{ |  p{0.3\linewidth} | p{0.3\linewidth} |  p{0.3\linewidth} |}
    \hline
    Positives & Deltas & Actions \\ \hline
    Our assigned team happens to be very synergistic. & We are more effective when we meet up in person, so we need to do that more frequently.
    & Once we receive our winter term schedules we will set a specific meeting time each week do work..  \\ \hline
    Our client has a lead developer who is more than happy to meet with our team approximately once a week. &
    Not all group members currently have access to the primary Github repository for the project. &
    We will communicate with the lead developer ASAP to gain access to the primary Github repository.  \\ \hline
    We turned in all of our projects on time and were met with high marks across the board. & 10C & 21C\\ \hline
    All group members do a great job of scheduling at times that work for everyone. &  & \\ \hline
    \end{tabular}
\end{center}

\end{document}

