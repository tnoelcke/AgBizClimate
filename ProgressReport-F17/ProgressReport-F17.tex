\documentclass[onecolumn, draftclsnofoot,10pt, compsoc]{article}
\usepackage{graphicx}
\usepackage{url}
\usepackage{lscape}
\usepackage{setspace}
\usepackage{parskip}
\usepackage{geometry}
\geometry{textheight=9.5in, textwidth=7in}

% 1. Fill in these details
\def \CapstoneTeamName{AgBizClimate}
\def \CapstoneTeamNumber{26}
\def \GroupMemberOne{	Thomas Noelcke}
\def \GroupMemberTwo{	Shane Barrantes}
\def \GroupMemberThree{	Shengpei Yuan}
\def \CapstoneProjectName{ Linking Seasonal Weather Data to AgBizClimate\texttrademark}
\def \CapstoneSponsorCompany{ Oregon State University}
\def \CapstoneSponsorPerson{ Clark Seavert}

% 2. Uncomment the appropriate line below so that the document type works
\def \DocType{		%Software Requirements Document
				%Requirements Document
				%Technology Review
				%Design Document
				Progress Report
				}

\newcommand{\NameSigPair}[1]{\par
\makebox[2.75in][r]{#1} \hfil 	\makebox[3.25in]{\makebox[2.25in]{\hrulefill} \hfill		\makebox[.75in]{\hrulefill}}
\par\vspace{-12pt} \textit{\tiny\noindent
\makebox[2.75in]{} \hfil		\makebox[3.25in]{\makebox[2.25in][r]{Signature} \hfill	\makebox[.75in][r]{Date}}}}
% 3. If the document is not to be signed, uncomment the RENEWcommand below
%\renewcommand{\NameSigPair}[1]{#1}

%%%%%%%%%%%%%%%%%%%%%%%%%%%%%%%%%%%%%%%
\begin{document}
\begin{titlepage}
    \pagenumbering{gobble}
    \begin{singlespace}
        \hfill
        % 4. If you have a logo, use this includegraphics command to put it on the coversheet.
        %\includegraphics[height=4cm]{CompanyLogo}
        \par\vspace{.2in}
        \centering
        \scshape{
            \huge CS Capstone \DocType \par
            {\large\today}\par
            \vspace{.5in}
            \textbf{\Huge\CapstoneProjectName}\par
            \vfill
            {\large Prepared for}\par
            \Huge \CapstoneSponsorCompany\par
            \vspace{5pt}
            {\Large\NameSigPair{\CapstoneSponsorPerson}\par}
            {\large Prepared by }\par
            Group\CapstoneTeamNumber\par
            % 5. comment out the line below this one if you do not wish to name your team
            %\CapstoneTeamName\par
            \vspace{5pt}
            {\Large
                \NameSigPair{\GroupMemberOne}\par
                \NameSigPair{\GroupMemberTwo}\par
                \NameSigPair{\GroupMemberThree}\par
            }
            \vspace{20pt}
        }
        \begin{abstract}
					In this document we will give an overview of the progress we have made this term on the \textit{AgBizClimate} project. In this document we will begin giving a brief introduction to the project. In this section we will discuss the purpose of the project, the scope of the project and give an overview of the project function. Then we will discuss the current state of the project. In this section we will talk about what we have done, what we have to do and show a week by week summery of progress. Next we will discuss items that are blocking our progress moving forward. Finally we will give a Retrospective of the last ten weeks.\\
        \end{abstract}
    \end{singlespace}
\end{titlepage}
\newpage
\pagenumbering{arabic}
\tableofcontents
% 7. uncomment this (if applicable). Consider adding a page break.
\listoffigures 
\newpage
%\listoftables
\clearpage

% 8. now you write!
%this section can likely be coppied from the design doc.
\section{Introduction}
	\subsection{Purpose}
		The purpose of this document is to describe the progress we have made so far on the \textit{AgBizClimate} project. In this document we will give a brief introduction to the \textit{AgBizCliamte} project. In this section we will discuss the purpose of the project. Additionally, we will also discuss the scope of the project and an overview of the project functions.\\
		This document is designed for the project owners. This document is also designed for the development team so we can evaluate our progress so far on this project. This project is also designed to fulfill the minimum requirements for the CS461 class for the OSU computer science program.\\
		
				\subsection{Overview}
			Seasonal climate is one of the essential factors that affects agricultural production. As a module of \textit{AgBiz Logic}, \textit{AgBizClimate} delivers essential information about climate change to farmers, and help professionals to develop management pathways that best fit their operations under a changing climate. This project aims to link the crucial seasonal climate data from the Northwest Climate Toolbox database to \textit{AgBiz Logic} so that it can provide changes in net returns of crop and livestock enterprises through powerful graphics and tables.\\

		\subsection{Scope}
			This project is a part of a much larger AgBiz Logic\texttrademark program. However, the purpose of this project is to add a short term climate tool to the \textit{AgBizClimate} module. This limits the scope of the project to the \textit{AgBizClimate} Module. Additionally, we will only be adding the short term climate data tool as the long term climate data tool already exists.\\

			Currently \textit{AgBizClimate} has a long-term climate tool but no such tool exists for short term climate data. We will implement a tool to extract short-term climate data from the Northwest Climate Toolbox database, display it to the user and allow the user to adjust crop and livestock yields or quality of products sold and, production inputs. Moreover, a landing tool will be developed to allow users to switch between short-term seasonal tool and long-term climate data tool.\\

\section{State of the Project}
	\subsection{Resolved Items}
	
	\subsection{ToDo's}
	
	\subsection{Blockers}
	
	\subsection{Weekly summary of progress}
	
		\subsubsection{Week 1}
		
		\subsubsection{Week 2}
		
		\subsubsection{Week 3}
		
		\subsubsection{Week 4}
		
		\subsubsection{Week 5}
		
		\subsubsection{Week 6}
		
		\subsubsection{Week 7}
		
		\subsubsection{Week 8}
		
		\subsubsection{Week 9}
		
		\subsubsection{Week 10}
		
		\subsubsection{Finals Week}
		
%this section will be a three column Positives, Deltas: Things that need to change, Action Column: things that will need to be implemented in order to create nescessary changes.
\section{Retrospective}



\end{document}

