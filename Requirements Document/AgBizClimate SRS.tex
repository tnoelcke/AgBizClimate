\documentclass[onecolumn, draftclsnofoot,10pt, compsoc]{IEEEtran}
\usepackage{graphicx}
\usepackage{url}
\usepackage{setspace}

\usepackage{geometry}
\geometry{textheight=9.5in, textwidth=7in}

% 1. Fill in these details
\def \CapstoneTeamName{AgBizClimate}
\def \CapstoneTeamNumber{26}
\def \GroupMemberOne{	Thomas Noelcke}
\def \GroupMemberTwo{	Shane Barrantes}
\def \GroupMemberThree{	Shengpei Yuan}
\def \CapstoneProjectName{ Linking Seasonal Weather Data to AgBizClimate\texttrademark}
\def \CapstoneSponsorCompany{ Oregon State University}
\def \CapstoneSponsorPerson{ Clark Seavert}

% 2. Uncomment the appropriate line below so that the document type works
\def \DocType{		Software Requirements Document
				%Requirements Document
				%Technology Review
				%Design Document
				%Progress Report
				}
			
\newcommand{\NameSigPair}[1]{\par
\makebox[2.75in][r]{#1} \hfil 	\makebox[3.25in]{\makebox[2.25in]{\hrulefill} \hfill		\makebox[.75in]{\hrulefill}}
\par\vspace{-12pt} \textit{\tiny\noindent
\makebox[2.75in]{} \hfil		\makebox[3.25in]{\makebox[2.25in][r]{Signature} \hfill	\makebox[.75in][r]{Date}}}}
% 3. If the document is not to be signed, uncomment the RENEWcommand below
%\renewcommand{\NameSigPair}[1]{#1}

%%%%%%%%%%%%%%%%%%%%%%%%%%%%%%%%%%%%%%%
\begin{document}
\begin{titlepage}
    \pagenumbering{gobble}
    \begin{singlespace}
    	\includegraphics[height=4cm]{coe_v_spot1}
        \hfill 
        % 4. If you have a logo, use this includegraphics command to put it on the coversheet.
        %\includegraphics[height=4cm]{CompanyLogo}   
        \par\vspace{.2in}
        \centering
        \scshape{
            \huge CS Capstone \DocType \par
            {\large\today}\par
            \vspace{.5in}
            \textbf{\Huge\CapstoneProjectName}\par
            \vfill
            {\large Prepared for}\par
            \Huge \CapstoneSponsorCompany\par
            \vspace{5pt}
            {\Large\NameSigPair{\CapstoneSponsorPerson}\par}
            {\large Prepared by }\par
            Group\CapstoneTeamNumber\par
            % 5. comment out the line below this one if you do not wish to name your team
            %\CapstoneTeamName\par 
            \vspace{5pt}
            {\Large
                \NameSigPair{\GroupMemberOne}\par
                \NameSigPair{\GroupMemberTwo}\par
                \NameSigPair{\GroupMemberThree}\par
            }
            \vspace{20pt}
        }
        \begin{abstract}   
				
        \end{abstract}     
    \end{singlespace}
\end{titlepage}
\newpage
\pagenumbering{arabic}
\tableofcontents
% 7. uncomment this (if applicable). Consider adding a page break.
%\listoffigures
%\listoftables
\clearpage

% 8. now you write!
\section{Introduction}
		\subsection{Purpose}
		This SRS describes the requirements and specifications of the AgBizClimate\texttrademark web application. This document will explain the functional features of this web application. This includes the interface details, design constraints and considerations such as performance characteristics. This SRS is intended out outline how we will proceed with the development of the AgBizClimate\texttrademark system.
		\end {Purpose}
		
		\subsection{Scope}
		\end{Scope}
		
		\subsection{Definitions, Acronyms and Abbreviations}
		
		\end{Definitions, Acronyms and Abbreviations}
		
		\subsection{References}
		
		\end{References}
		
		\subsection{Overview}
		
		\end{Overview}
\end {Introduction}
\section{Overall Description}
	\subsection{Product perspective}
		Here we will talk about the product in a general sense and talk about what it is supposed to do from a high level.
	\end
	
	\subsection {Product Functions}
		Describe specifically what the product is supposed to do.
	\end
	
	\subsection{User Characteristics}
		Talk about who will be using our product and briefly describe them.
	\end{User Characteristics}
	
	\subsection{Constraints}
		
	\end{Constraints}
	
	\subsection {Assumptions and Dependencies}
	
	\end{Assumptions and Dependencies}
	
\end {Overall Description}

\section{Specific Requirements}
This section contains all of the functional and quality requirements of the AgBizClimate System. We will give detailed description of what's being added to the AgBizLogic system along with the features we will implement.
    \subsection{External Interface Requirements}
		This section provides a detailed description of all inputs into and outputs from the AgBizClimate system. This section will also provide descriptions for the hardware, software and communication interfaces. Additionally we will provide detailed prototypes for the user interface.
        \subsubsection{User Interface}
        \end{User Interface}

        \subsubsection{Hardware Interfaces}

        \end{Hardware Interface}

        \subsubsection{Software Interface}
        
        \end{Software Interface}

        \subsubsection{Communications Interfaces}

        \end{Communications Interfaces}
    \end{External Interface Requirements}

    \subsection{User Stories}

    \end{User Stories}
			%Decided to go with functional requirements instead of user stories because
			%it will make our Gantt Chart much easier to draft.
			\subsubsection{Functional Requirements For API}
					\paragraph{Functional Requirement 1.1}
						Get User data
					\end{Functional Requirement 1.1}

					\paragrph{Functional Requirement 1.2}
						Transform Location
					\end{Functional Requirement 1.2}

					\paragrph{Functional Requirement 1.3}
						Authentication with NWCTB API
					\end{Functional Requirement 1.3}

					\paragraph{Functional Requirement 1.4}
						Request Data
					\end{Functional Requirement 1.4}
					
					\paragraph{Functional Requirement 1.5}
						Take data from FR 1.4 and place it in a json object with useful labels.
					\end{Functional Requirement 1.5}
						Send Data to the front end through an HTTP request so it can be Plotted.
					\paragraph{Plot The resulting data}

					\end{Plot The Resulting data}
			\end{Functional Requirements For API}
			
			\subsubsection{Functional Requirements For Front End}
				\paragrph{Functional Requirement 2.1}
					In this section I will describe the landing page that will allow the user to switch between long term and short term climate data.
				\end{Functional Requirement 2.1}

				\paragraph{Functional Requirement 2.2}
					In this section I will talk about how the user will enter in their location by county and state and will click submit. This will fire a HTTP request for the data.
				\end{Functional Requirement 2.2}
				
				\paragraph{Functional Requirement 2.3}
					In this section I will talk about how the front end will receive the data from the API and will then plot it.
				\end{Functional Requirement 2.3}
				
				\paragraph{Functional Requirement 2.4}
					In this section i will talk about how the user will then enter what percentage they think this will effect the yield of their crop.
				\end{Functional Requirement 2.4}
				
				\paragraph{Functional Requirement 2.5}
					In this section i will talk about how we will take the user input and pass it
					to the existing AgBizClimate budget page.
				\end{Functional Requirement 2.5}
			\end {Functional Requirements for Front End}
    \end{Functional Requirements}

    \subsection{Performance Requirements}

    \end{Performance Requirements}

    \subsection{Design Constraints}

    \end{Design Constraints}

    \subsection{Software System Attributes}

    \end{Software System Attributes}

    \subsection{Other Requirements}

    \end{Other Requirements}
\end {Specific Requirements}



\section{Gantt Chart}
	more or less a burn down chart
\end {Gantt Chart}
		

\end{document}